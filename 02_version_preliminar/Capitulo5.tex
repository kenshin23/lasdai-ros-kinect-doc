\chapter{Generación de Mapas a través de RTAB-Map}

En este capítulo se describe el software RTAB-Map, sus características, su proceso de funcionamiento, su instalación y uso en el entorno de desarrollo en sus presentaciones como software independiente o como módulo integrado a ROS y por último, la generación de un mapa de entorno a través del mismo.

\section{Definición}

RTAB-Map (\textit{Real-Time Appearance-Based Mapping} o Cartografía en Tiempo Real Basada en Apariencias) es una aproximación a SLAM mediante Grafo RGB-D basado en un detector de cierre de lazo Bayesiano global. El detector de cierre de lazo usa un modelo de bolsa de palabras para determinar la probabilidad de que una nueva imagen provenga de una ubicación anterior o nueva. Cuando una hipótesis de cierre de lazo es aceptada, una nueva restricción es agregada al grafo del mapa, y un optimizador de grafo minimiza los errores en el mapa.

Un enfoque de manejo de memoria es utilizado para limitar el número de ubicaciones utilizadas para la detección de cierre de lazos y la optimización del grafo, para que las restricciones de tiempo real en entornos de gran escala sean siempre respetadas. RTAB-Map puede ser utilizado por si solo con un Kinect o cámara estéreo operado a mano para obtener cartografía RGB-D de 6 grados de libertad, o en un robot equipado con un medidor de distancias láser para cartografía de 3 grados de libertad. \cite{rtabmaphome}

\section{Instalación}

RTAB-Map soporta instalaciones en Linux, OS X y Windows y puede funcionar de dos maneras: como software independiente (no necesita otros paquetes de software aparte de él mismo y de los controladores de las cámaras a usar) o como un módulo de ROS, en cuyo caso lógicamente requiere que ROS esté instalado de antemano (y su compatibilidad se reduce a la misma de ROS).

\subsection{Como software independiente}

Para instalar el software de forma independiente, tomaremos las instrucciones correspondientes a la instalación en Ubuntu debido a que es el entorno de desarrollo utilizado. Estas instrucciones son tomadas del repositorio del proyecto en Github, a través de la dirección \url{https://github.com/introlab/rtabmap/wiki/Installation}:

\begin{itemize}
	\item Con ROS ya instalado en el sistema:

	Si ya está instalado ROS en el sistema (como es el caso en el desarrollo del proyecto), ya algunas dependencias estarán instaladas:

	Dependencias según versión de ROS:
	Indigo:
	\begin{blackcodebox}
	\begin{lstlisting}[language=bash]
sudo apt-get install libsqlite3-dev libpcl-1.7-all libfreenect-dev libopencv-dev
	\end{lstlisting}
	\end{blackcodebox}
	Hydro:
	\begin{blackcodebox}
	\begin{lstlisting}[language=bash]
sudo apt-get install libsqlite3-dev libpcl-1.7-all ros-hydro-libfreenect ros-hydro-opencv2
	\end{lstlisting}
	\end{blackcodebox}

	Descargue el código fuente de RTAB-Map desde Github:
	\begin{blackcodebox}
	\begin{lstlisting}[language=bash]
git clone https://github.com/introlab/rtabmap.git rtabmap
cd rtabmap/build
cmake ..
make -j4
make install
	\end{lstlisting}
	\end{blackcodebox}

	Ya puede ejecutar la aplicación (llamada ``rtabmap'').

	\item Si ROS no está instalado:

	Dependencias del sistema:
	\begin{blackcodebox}
	\begin{lstlisting}[language=bash]
sudo apt-get install libsqlite3-dev libpcl-1.7-all libopencv-dev
	\end{lstlisting}
	\end{blackcodebox}

	Para instalar libpcl-1.7-all, es posible que deba agregar los repositorios de ROS (en este caso particular, de la distribución de ROS compatible con la distribución de Ubuntu) realizando los siguientes pasos:
	\begin{blackcodebox}
	\begin{lstlisting}[language=bash]
sudo sh -c 'echo "deb http://packages.ros.org/ros/ubuntu $(lsb_release -sc) main" > /etc/apt/sources.list.d/ros-latest.list'
wget http://packages.ros.org/ros.key -O - | sudo apt-key add -
sudo apt-get update
	\end{lstlisting}
	\end{blackcodebox}

	Si desea habilitar las características SURF/SIFT (SURF: \textit{Speeded-Up Robust Features} -- Características Robustas Aceleradas; SIFT: \textit{Scale-Invariant Feature Transform} -- Transformación de Características Invariantes en Escala) en RTAB-Map, deberá descargar y generar OpenCV desde el código fuente para tener acceso al módulo no-libre/privativo:
	\begin{blackcodebox}
	\begin{lstlisting}[language=bash]
cd opencv
mkdir build
cd build
cmake -DCMAKE_BUILD_TYPE=Release ..
make -j4
sudo make install
	\end{lstlisting}
	\end{blackcodebox}

	Descargue el código fuente de RTAB-Map desde Github: obtenga la última versión o el código fuente actual
	\begin{blackcodebox}
	\begin{lstlisting}[language=bash]
git clone https://github.com/introlab/rtabmap.git rtabmap
cd rtabmap/build
cmake ..
make -j4
sudo make install
	\end{lstlisting}
	\end{blackcodebox}

	Ya puede ejecutar la aplicación (llamada ``rtabmap'').

	\item Actualizar el código fuente de RTAB-Map

	Si desea incorporar los últimos cambios después de realizar el ``git clone'' puede actualizarlo de la siguiente forma:
	\begin{blackcodebox}
	\begin{lstlisting}[language=bash]
cd rtabmap
git pull origin master
cd build
cmake ..
make -j4
sudo make install
	\end{lstlisting}
	\end{blackcodebox}
\end{itemize}

\subsection{Como módulo de ROS}

Ubuntu cuenta con binarios para las versiones Hydro e Indigo; basta con ejecutar el siguiente comando, según sea la distribución de ROS:

\noindent ROS Hydro:
\begin{blackcodebox}
\begin{lstlisting}[language=bash]
sudo apt-get install ros-hydro-rtabmap-ros
\end{lstlisting}
\end{blackcodebox}

\noindent ROS Indigo:
\begin{blackcodebox}
\begin{lstlisting}[language=bash]
sudo apt-get install ros-indigo-rtabmap-ros
\end{lstlisting}
\end{blackcodebox}

Si se desea instalar desde fuente, el proceso (detallado en la página \url{https://github.com/introlab/rtabmap_ros#rtabmap_ros}) conlleva tener conocimiento del espacio de trabajo (\textit{workspace}) de ROS, y la instalación desde fuente de la librería OpenCV. También se asume que se ha configurado el espacio de trabajo en el directorio \url{~/catkin_ws} y que el archivo \url{~/.bashrc} contiene lo siguiente:

\noindent ROS Hydro:
\begin{blackcodebox}
\begin{lstlisting}[language=bash]
source /opt/ros/hydro/setup.bash
source ~/catkin_ws/devel/setup.bash
\end{lstlisting}
\end{blackcodebox}

\noindent ROS Indigo:
\begin{blackcodebox}
\begin{lstlisting}[language=bash]
source /opt/ros/indigo/setup.bash
source ~/catkin_ws/devel/setup.bash
\end{lstlisting}
\end{blackcodebox}

Luego, se procede a descargar el código fuente de RTAB-Map desde Github (\textbf{NOTA:} No descargar dentro del espacio de trabajo) e instalarlo dentro del directorio \url{devel} en el espacio de trabajo, ejecutando lo siguiente:
\begin{blackcodebox}
\begin{lstlisting}[language=bash]
cd ~
git clone https://github.com/introlab/rtabmap.git rtabmap
cd rtabmap/build
cmake -DCMAKE_INSTALL_PREFIX=~/catkin_ws/devel ..
make -j4
sudo make install
\end{lstlisting}
\end{blackcodebox}

Ahora puede instalar el ros-pkg de RTAB-Map dentro del directorio \url{src} del espacio de trabajo Catkin:
\begin{blackcodebox}
\begin{lstlisting}[language=bash]
cd ~/catkin_ws
git clone https://github.com/introlab/rtabmap_ros.git src/rtabmap_ros
catkin_make
\end{lstlisting}
\end{blackcodebox}

\section{Generación de Mapas de Entorno}

---Llenar preámbulo---

RTAB-Map comenzará a capturar una nube de puntos en 3D mientras detecta cierres de lazo -- es decir, se lleva a cabo la detección en las ``imágenes'' (ya que no se captura una imagen en sí) de lugares ya visitados previamente, tras lo cual se realizan correcciones en las estimaciones pasadas. Una vez terminado, se puede exportar la nube de puntos capturada a distintos formatos (PCD, que es un formato que contiene datos de nube de puntos \cite{pcdformat} y PLY, que es un formato de polígonos, que guardan datos tridimensionales), para ser procesada posteriormente de ser necesario.

\subsection{Requerimientos}

---Llenar.---

\subsection{Procedimiento}

---Llenar.---

\subsection{Pruebas}

---Llenar.---

\subsubsection{Generación de mapas 3D}

---Llenar.---

\subsubsection{Generación de mapas 2D}

---Llenar.---

\paragraph{Selección de altura mínima y máxima para generación del mapa 2D}

---Llenar.---