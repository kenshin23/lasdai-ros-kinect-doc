\begin{displayquote}``Somos como enanos a los hombros de gigantes. Podemos ver más, y más lejos que ellos, no porque la agudeza de nuestra vista ni por la altura de nuestro cuerpo, sino porque somos levantados por su gran altura.'' (Bernardo de Chartres)
\end{displayquote}

\begin{displayquote}
``Si he visto más lejos es porque estoy sentado sobre los hombros de gigantes.'' (Isaac Newton, entre otros)
\end{displayquote}

Este proyecto de grado simboliza la culminación de un largo y árduo camino tras el cual doy fin a una etapa de vida e inicio otra, oficialmente como Ingeniero de Sistemas, pero más allá de eso, como profesional en eterna formación y aprendizaje. No habría podido llegar acá sin la ayuda y guía de las siguientes personas (y de muchas otras en mis recuerdos), para las cuales nunca bastarán las palabras de agradecimiento que pueda tener.

\vspace{4mm}

Sin orden particular, aunque parezca:

\vspace{4mm}

A mi Padre Celestial. A mis santos, mis ``viejos'', mi ángel de la guarda. Gracias por sus bendiciones y cuidados, gracias por siempre mantenerme en el camino correcto y resguardarme de todo lo malo.
\par
A mis padres, porque, sin duda alguna (y muy literalmente) no estaría aquí hoy de no ser por ustedes. Por enseñarme que ``rendirse'' no es un verbo que puedo ni debo conjugar en primera persona, por los valores que inculcaron en mi y porque de ustedes nunca faltó el apoyo que necesitaba para levantarme tras los inevitables tropiezos que he dado al andar en mi camino. Soy lo que soy gracias a ustedes. Mis logros son suyos. A ustedes, siempre, mis eternas gracias y todo el amor que tengo para darles.
\par
A mis hermanos, porque de ustedes he descubierto que si lo imaginas, lo haces posible. Han sido mis guías en grandiosos paisajes y aunque les llevo ventaja en edad, he confiado mis pasos detrás de ustedes sin dudar por un instante. Gracias por ser. Gracias por estar.
\par
A toda mi familia, que siempre me ha apoyado y apoya en cada paso que doy, que son quienes me empujan a ser cada día mejor y a luchar por los sueños, porque me han enseñado y demostrado que ni siquiera el cielo es el límite y que si lo quiero, lo puedo.
\par
A la ilustre Universidad de Los Andes, de la cual orgullosamente formo parte y a mis profesores a lo largo de mi carrera profesional, porque independientemente de sus acciones u omisiones, han formado -cual cincel a bloque de piedra- al ingeniero que hoy ven ante ustedes. A todos los que tomaron en cuenta mi condición particular y tendieron una mano amiga o dieron invaluables consejos, e incluso a los que no, les doy mis más sinceras gracias. Menciones especiales a los profesores Demián Gutierrez, Oswaldo Ramirez, Fabiola Díaz, Luz Marina Pereira, y por último pero no por ello menos importante, a mi tutor, Rafael Rivas. Gracias por no permitirme desistir.
\par
A mi amada Sandra. Palabras faltan para decirte lo mucho que agradezco de corazón tu apoyo y ánimo cuando más lo necesitaba, porque me brindaste luz cuando sólo parecía encontrar oscuridad. Te amo. No habría podido llegar aquí sin ti.
\par
A mis amigos y compañeros de estudio, que hicieron de esta etapa universitaria un gran momento para recordar y que compartieron conmigo sus triunfos, derrotas y aprendizajes, mil gracias. A riesgo siempre de omitir (sin intención) algún nombre, pero no por ello apreciarles y quererles menos: Armando, Carla, Dario, Fernando, Gerardo, Idaí, Iramsaby, Jesús, Joel, Jorge, Junior, Karla, Liliberth, Luis, Manuel, Stefan, Syra, Totti, Vladimir.
\par
A todos, los que hoy comparten conmigo este logro, gracias totales.